\documentclass{article}
\usepackage[utf8]{inputenc}
\usepackage{amsmath}

\begin{document}

\section*{Exercice : Modèle AS-AD et ZLB}

On considère une version simplifiée du modèle AS/AD du cours:
\[ \text{MP}: r=\gamma \left(\pi-\pi^{\star}\right)\]
\[ \text{IS}: y=-\sigma r + \theta \]
\[ \text{PC}: y=\kappa (\pi-\pi^{\star})\]

où $r$ est le taux d'intérêt réel, où $\pi$ est l'inflation, $\pi^{\star}$ la cible d'inflation et où l'output gap ("pib potentiel") est noté $y$. La variable $\theta$ est un choc de demande et $\kappa$, $\sigma$, $\gamma$ sont tous des paramètres positifs.

Initialement, l'économie se trouve à l'équilibre de plein emploi.
Puis un gouvernement nouvellement élu décide de réduire drastiquement les dépenses publiques.


\begin{enumerate}
    \item De quel type de choc s'agit-il ? Quel est l'impact sur la courbe AD ? Dans quel sens la banque centrale ajuste-t-elle ses taux d'intérêt ?
    \item Représentez graphiquement l'impact de ce choc à court terme.
    \item On suppose maintenant que la banque centrale ne peut pas baisser le 
    taux réel en dessous d'un certaine borne inférieure $\underbar{r}$\footnote{Lorsque cette borne inférieure est zéro, on l'appelle en anglais "zero lower bound" (ZLB).} On supposera que la courbe MP devient $r=max\left(\underbar{r},\gamma \left(\pi-\pi^{\star}\right)\right)$. Analyser alors l'effet du choc s'il est assez grand pour atteindre cette borne inférieure.
\end{enumerate}
    

\section*{Exercice : Consommation intertemporelle, investissement immobilier et utilité du logement}

Un agent économique vit pendant deux périodes $(t = 1, 2)$. Il dispose d'un revenu initial $Y_1$ à la période 1 et d'un revenu anticipé $Y_2$ à la période 2.

L'agent peut investir dans un actif immobilier dont le prix à la période 1 est $P_1$ et qui est revendu à la période 2 au prix (parfaitement anticipé) $P_2$. Cet actif procure un service de logement qui augmente directement son utilité.

L'agent peut aussi emprunter ou épargner au taux réel $r$.

Il maximise la fonction d'utilité intertemporelle suivante :
\[
U(C_1, C_2, H) = \log(C_1) + \beta \log(C_2) + \gamma \log(H)
\]
avec $0 < \beta < 1$ et $\gamma > 0$, où $C_1$ et $C_2$ représentent les niveaux de consommation aux périodes 1 et 2, et $H$ la quantité d'actif immobilier acquise en période 1.

On suppose qu'aucune contrainte de liquidité n'est imposée à l'agent, c'est-à-dire qu'il peut emprunter ou épargner sans limite. De même la quantité d'actifs immobiliers sur la marché, n'est pas limitée de sorte que le choix optimal de $H$ n'est pas contraint.

\subsection*{Questions et corrections}

\begin{enumerate}
    \item Écrivez la contrainte budgétaire intertemporelle.

% \textit{Correction :}
% \[
% C_1 + P_1 H + \frac{C_2}{1+r} = Y_1 + \frac{Y_2 + P_2 H}{1+r}
% \]

%     \item \textbf{Déterminez analytiquement les conditions d'optimalité.}

% \textit{Correction :}
% Les conditions de premier ordre sont :
% \[
% \frac{1}{C_1} = \frac{\beta(1+r)}{C_2}, \quad \frac{1}{C_1} P_1 = \frac{\gamma}{H} + \frac{P_2}{C_1 (1+r)}
% \]

    \item Montrez que la condition d'arbitrage entre immobilier et actif sans risque peut s'écrire:
\[
r = \frac{P_2 - P_1}{P_1} + \frac{\gamma C_1 (1+r)}{P_1 H}
\]
% \textit{Correction :}
% La condition d'arbitrage s'écrit :
% \[
% r = \frac{P_2 - P_1}{P_1} + \frac{\gamma C_1}{P_1 H}
% \]
% Elle signifie que l'agent équilibre le rendement financier ajusté par l'utilité directe du logement avec celui de l'actif sans risque.

    \item Quel est l'effet d'une hausse anticipée du prix immobilier futur $P_2$ ?

% \textit{Correction :}
% Une hausse de $P_2$ rend l'immobilier plus attractif, ce qui pousse l'agent à augmenter l'investissement immobilier à la période 1, réduisant ainsi probablement sa consommation actuelle.

    \item Quel est l'effet d'une hausse du paramètre $\gamma$ ?

% \textit{Correction :}
% Si $\gamma$ augmente, l'utilité directe de l'immobilier augmente. L'agent investira davantage dans l'immobilier au détriment probablement de sa consommation actuelle.

\end{enumerate}

\end{document}
