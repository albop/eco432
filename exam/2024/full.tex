% Options for packages loaded elsewhere
\PassOptionsToPackage{unicode}{hyperref}
\PassOptionsToPackage{hyphens}{url}
\PassOptionsToPackage{dvipsnames,svgnames,x11names}{xcolor}
%
\documentclass[
  letterpaper,
  DIV=11,
  numbers=noendperiod]{scrartcl}

\usepackage{amsmath,amssymb}
\usepackage{iftex}
\ifPDFTeX
  \usepackage[T1]{fontenc}
  \usepackage[utf8]{inputenc}
  \usepackage{textcomp} % provide euro and other symbols
\else % if luatex or xetex
  \usepackage{unicode-math}
  \defaultfontfeatures{Scale=MatchLowercase}
  \defaultfontfeatures[\rmfamily]{Ligatures=TeX,Scale=1}
\fi
\usepackage{lmodern}
\ifPDFTeX\else  
    % xetex/luatex font selection
\fi
% Use upquote if available, for straight quotes in verbatim environments
\IfFileExists{upquote.sty}{\usepackage{upquote}}{}
\IfFileExists{microtype.sty}{% use microtype if available
  \usepackage[]{microtype}
  \UseMicrotypeSet[protrusion]{basicmath} % disable protrusion for tt fonts
}{}
\makeatletter
\@ifundefined{KOMAClassName}{% if non-KOMA class
  \IfFileExists{parskip.sty}{%
    \usepackage{parskip}
  }{% else
    \setlength{\parindent}{0pt}
    \setlength{\parskip}{6pt plus 2pt minus 1pt}}
}{% if KOMA class
  \KOMAoptions{parskip=half}}
\makeatother
\usepackage{xcolor}
\setlength{\emergencystretch}{3em} % prevent overfull lines
\setcounter{secnumdepth}{-\maxdimen} % remove section numbering
% Make \paragraph and \subparagraph free-standing
\ifx\paragraph\undefined\else
  \let\oldparagraph\paragraph
  \renewcommand{\paragraph}[1]{\oldparagraph{#1}\mbox{}}
\fi
\ifx\subparagraph\undefined\else
  \let\oldsubparagraph\subparagraph
  \renewcommand{\subparagraph}[1]{\oldsubparagraph{#1}\mbox{}}
\fi


\providecommand{\tightlist}{%
  \setlength{\itemsep}{0pt}\setlength{\parskip}{0pt}}\usepackage{longtable,booktabs,array}
\usepackage{calc} % for calculating minipage widths
% Correct order of tables after \paragraph or \subparagraph
\usepackage{etoolbox}
\makeatletter
\patchcmd\longtable{\par}{\if@noskipsec\mbox{}\fi\par}{}{}
\makeatother
% Allow footnotes in longtable head/foot
\IfFileExists{footnotehyper.sty}{\usepackage{footnotehyper}}{\usepackage{footnote}}
\makesavenoteenv{longtable}
\usepackage{graphicx}
\makeatletter
\def\maxwidth{\ifdim\Gin@nat@width>\linewidth\linewidth\else\Gin@nat@width\fi}
\def\maxheight{\ifdim\Gin@nat@height>\textheight\textheight\else\Gin@nat@height\fi}
\makeatother
% Scale images if necessary, so that they will not overflow the page
% margins by default, and it is still possible to overwrite the defaults
% using explicit options in \includegraphics[width, height, ...]{}
\setkeys{Gin}{width=\maxwidth,height=\maxheight,keepaspectratio}
% Set default figure placement to htbp
\makeatletter
\def\fps@figure{htbp}
\makeatother

\KOMAoption{captions}{tableheading}
\makeatletter
\@ifpackageloaded{tcolorbox}{}{\usepackage[skins,breakable]{tcolorbox}}
\@ifpackageloaded{fontawesome5}{}{\usepackage{fontawesome5}}
\definecolor{quarto-callout-color}{HTML}{909090}
\definecolor{quarto-callout-note-color}{HTML}{0758E5}
\definecolor{quarto-callout-important-color}{HTML}{CC1914}
\definecolor{quarto-callout-warning-color}{HTML}{EB9113}
\definecolor{quarto-callout-tip-color}{HTML}{00A047}
\definecolor{quarto-callout-caution-color}{HTML}{FC5300}
\definecolor{quarto-callout-color-frame}{HTML}{acacac}
\definecolor{quarto-callout-note-color-frame}{HTML}{4582ec}
\definecolor{quarto-callout-important-color-frame}{HTML}{d9534f}
\definecolor{quarto-callout-warning-color-frame}{HTML}{f0ad4e}
\definecolor{quarto-callout-tip-color-frame}{HTML}{02b875}
\definecolor{quarto-callout-caution-color-frame}{HTML}{fd7e14}
\makeatother
\makeatletter
\@ifpackageloaded{caption}{}{\usepackage{caption}}
\AtBeginDocument{%
\ifdefined\contentsname
  \renewcommand*\contentsname{Table of contents}
\else
  \newcommand\contentsname{Table of contents}
\fi
\ifdefined\listfigurename
  \renewcommand*\listfigurename{List of Figures}
\else
  \newcommand\listfigurename{List of Figures}
\fi
\ifdefined\listtablename
  \renewcommand*\listtablename{List of Tables}
\else
  \newcommand\listtablename{List of Tables}
\fi
\ifdefined\figurename
  \renewcommand*\figurename{Figure}
\else
  \newcommand\figurename{Figure}
\fi
\ifdefined\tablename
  \renewcommand*\tablename{Table}
\else
  \newcommand\tablename{Table}
\fi
}
\@ifpackageloaded{float}{}{\usepackage{float}}
\floatstyle{ruled}
\@ifundefined{c@chapter}{\newfloat{codelisting}{h}{lop}}{\newfloat{codelisting}{h}{lop}[chapter]}
\floatname{codelisting}{Listing}
\newcommand*\listoflistings{\listof{codelisting}{List of Listings}}
\makeatother
\makeatletter
\makeatother
\makeatletter
\@ifpackageloaded{caption}{}{\usepackage{caption}}
\@ifpackageloaded{subcaption}{}{\usepackage{subcaption}}
\makeatother
\ifLuaTeX
  \usepackage{selnolig}  % disable illegal ligatures
\fi
\usepackage{bookmark}

\IfFileExists{xurl.sty}{\usepackage{xurl}}{} % add URL line breaks if available
\urlstyle{same} % disable monospaced font for URLs
\hypersetup{
  pdftitle={Examen (Partie 1)},
  colorlinks=true,
  linkcolor={blue},
  filecolor={Maroon},
  citecolor={Blue},
  urlcolor={Blue},
  pdfcreator={LaTeX via pandoc}}

\title{Examen (Partie 1)}
\author{}
\date{}

\begin{document}
\maketitle

\subsection{Questions de cours}\label{questions-de-cours}

Choisir l'unique bonne réponse:

1/ Laquelles des assertions suivantes est fausse:

\begin{enumerate}
\def\labelenumi{\alph{enumi}.}
\tightlist
\item
  la propension marginale à consommer des consommateurs est comprise
  entre 0 et 1
\item
  la consommation des consommateurs ricardiens réagit au taux d'intérêt
  réel
\item
  si tous les consommateurs sont keynésiens, une baisse du taux
  d'intérêt réel ne stimule pas la demande agrégée
\item
  la banque centrale stabilise la demande en influant sur le taux
  d'intérêt réel
\end{enumerate}

2/ A la suite d'un choc inconnu, on a observé une baisse de la
production accompagnée d'une augmentation de l'inflation. Après
plusieurs périodes, la production est revenue à son niveau d'origine
mais l'inflation est restée à un niveau plus haut. Quel type d'événement
est compatible avec cette observation:

\begin{enumerate}
\def\labelenumi{\alph{enumi}.}
\tightlist
\item
  Un choc négatif persistent de la production et un choc négatif
  temporaire de la demande
\item
  Un choc négatif temporaire de la production et un choc postif
  persistent de la demande
\item
  Un choc positif temporaire de la production et un choc négatif
  persistent de la demande
\item
  Un choc positif persistent de la production et un choc négatif
  temporaire de la demande
\end{enumerate}

\subsection{Équilibre à long terme et marché du
travail.}\label{uxe9quilibre-uxe0-long-terme-et-marchuxe9-du-travail.}

On suppose que les firmes produisent avec une technologie linéaire
\(Y_t = L_t Z_t\) où \(L_t\) est le nombre d'heures travaillées, et
\(Z_t\) un choc de productivité. Le salaire horaire est \(W_t\).

1/ Quel est le coût marginal de la production?

Les travailleurs maximisent chaque période une fonction d'utilité
\(V(C_t,L_t) = \frac{{C_t}^{1-\sigma}}{1-\sigma} - \frac{1}{\xi}{L_t}^{\xi}\)
où \(C_t\) est la consommation et \(L_t\) le nombre d'heures travaillées
et \(\xi\) un paramètre positif. On note \(P_t\) le niveau des prix.

2/ Écrire la contrainte de budget intratemporelle des travailleurs et
déterminer leur offre de travail à l'équilibre.

3/ Quel est l'équilibre sur le marché du travail? Représentation
graphique.

On suppose maintenant que les firmes fixent leur prix optimal
\(P^{\star}_t\) en intégrant une marge \(\mu\) sur le coût marginal.

4/ En supposant les prix parfaitement flexibles calculer l'équilibre de
long terme pour les différentes variables macroéconomiques. Commenter
l'effet de la productivité \(Z\) et du taux de marge \(\mu\) sur \(Y\)
et \(L\).

\subsection{Principe de Taylor}\label{principe-de-taylor}

On considère ici une économie loglinéarisée caractérisée par les courbes
IS et PC suivantes:

\begin{equation}\phantomsection\label{eq-is}{y_t = y_{t+1} - \sigma \left( i_t - \pi_{t+1} \right) + e^{\pi}_t}\end{equation}
\begin{equation}\phantomsection\label{eq-pc}{\pi_t = \kappa (y_t - e^y_t)  + \beta \pi_{t+1}}\end{equation}

où \(\pi_t\) dénote l'inflation, \(y_t\) la production, \(i_t\) le taux
d'intérêt et où \(\sigma\) et \(\kappa\) sont des paramètres réels
positifs et où \(\beta \in ]0,1[\) est le facteur d'escompte.

Les variables \(e^{\pi}_t\) et \(e^{y}_t\) sont respectivement des chocs
de demande et d'offre.\footnote{Ici, la courbe de Philips provient de la
  fixation des prix par des entreprises en compétitions monopolistiques,
  qui optimisent leur profits futurs plutôt qu'instantané. On parle de
  courbe de Philips augmentée par les anticipations.}. Ils sont pris
comme exogènes et on les suppose bornés. On suppose qu'il n'y a pas
d'incertitude sur la valeur des chocs futurs, de sorte qu'on peut
omettre les symboles d'espérance et considérer toutes leurs valeurs
comme connues.

La banque centrale suit une règle pour fixer son taux d'intérêt:
\begin{equation}\phantomsection\label{eq-taylor}{i_t = i^{\star} + \varphi_y (y_t - e^{\pi}_t) + \varphi_\pi (\pi_t - \pi^{\star})}\end{equation}

avec la cible d'inflation égale au taux d'intérêt cible:
\(i^{\star}=\pi^{\star}\).

On dit qu'une règle de Taylor satisfait le principe de Taylor, si en
réponse à un choc de demande permanent ayant pour effet d'augmenter
l'inflation d'1\%, la banque centrale augmente le taux d'intérêt de plus
d'1\%.

\begin{enumerate}
\def\labelenumi{\arabic{enumi}.}
\tightlist
\item
  Définir deux matrices A, B telles que:
\end{enumerate}

\[z_{t+1} = A z_t + B e_t\]

où \(z_t=(\pi_t, y_t)\) et \(e_t = (e^{\pi}_t, e^{y}_t)\)

\begin{enumerate}
\def\labelenumi{\arabic{enumi}.}
\setcounter{enumi}{1}
\tightlist
\item
  Montrer que les niveaux d'inflation et de production sont uniquement
  déterminés à toutes les dates \(t\geq 0\) si
  \[\varphi_{\pi} + \frac{1-\beta}{\kappa} \varphi_{y}> 1\] Cela
  correspond-il a une banque centrale plus active ou plus passive
  vis-à-vis de l'inflation? Comparer avec le principe de Taylor.
\end{enumerate}



\end{document}
