%2multibyte Version: 5.50.0.2960 CodePage: 1252

\documentclass[a4paper,11pt]{article}
%%%%%%%%%%%%%%%%%%%%%%%%%%%%%%%%%%%%%%%%%%%%%%%%%%%%%%%%%%%%%%%%%%%%%%%%%%%%%%%%%%%%%%%%%%%%%%%%%%%%%%%%%%%%%%%%%%%%%%%%%%%%%%%%%%%%%%%%%%%%%%%%%%%%%%%%%%%%%%%%%%%%%%%%%%%%%%%%%%%%%%%%%%%%%%%%%%%%%%%%%%%%%%%%%%%%%%%%%%%%%%%%%%%%%%%%%%%%%%%%%%%%%%%%%%%%
\usepackage{amsfonts}
\usepackage{graphicx}
\usepackage{amsmath}

\setcounter{MaxMatrixCols}{10}
%TCIDATA{OutputFilter=LATEX.DLL}
%TCIDATA{Version=5.50.0.2960}
%TCIDATA{Codepage=1252}
%TCIDATA{<META NAME="SaveForMode" CONTENT="1">}
%TCIDATA{BibliographyScheme=Manual}
%TCIDATA{Created=Wed Apr 05 06:17:18 2000}
%TCIDATA{LastRevised=Tuesday, October 23, 2018 17:59:03}
%TCIDATA{<META NAME="GraphicsSave" CONTENT="32">}
%TCIDATA{<META NAME="DocumentShell" CONTENT="General\Blank Document">}
%TCIDATA{Language=French}
%TCIDATA{CSTFile=LaTeX article (bright).cst}

\newtheorem{theorem}{Theorem}
\newtheorem{acknowledgement}[theorem]{Acknowledgement}
\newtheorem{algorithm}[theorem]{Algorithm}
\newtheorem{axiom}[theorem]{Axiom}
\newtheorem{case}[theorem]{Case}
\newtheorem{claim}[theorem]{Claim}
\newtheorem{conclusion}[theorem]{Conclusion}
\newtheorem{condition}[theorem]{Condition}
\newtheorem{conjecture}[theorem]{Conjecture}
\newtheorem{corollary}[theorem]{Corollary}
\newtheorem{criterion}[theorem]{Criterion}
\newtheorem{definition}[theorem]{Definition}
\newtheorem{example}[theorem]{Example}
\newtheorem{exercise}[theorem]{Exercise}
\newtheorem{lemma}[theorem]{Lemma}
\newtheorem{notation}[theorem]{Notation}
\newtheorem{problem}[theorem]{Problem}
\newtheorem{proposition}[theorem]{Proposition}
\newtheorem{remark}[theorem]{Remark}
\newtheorem{solution}[theorem]{Solution}
\newtheorem{summary}[theorem]{Summary}
\newenvironment{proof}[1][Proof]{\textbf{#1.} }{\ \rule{0.5em}{0.5em}}
%\input{tcilatex}
\oddsidemargin 0pt
\evensidemargin 0pt
\setlength\textwidth{17.5cm}
\setlength{\topmargin}{-2cm}
\setlength{\oddsidemargin}{-1cm}
\setlength\textheight{25cm}

\begin{document}


\begin{center}
\textbf{Ecole Polytechnique}

\bigskip

\textbf{Eco 432 - Macro\'{e}conomie}

\bigskip

\textbf{PC 4. L'effet macro\'{e}conomique de la d\'{e}pense publique}
\end{center}

\bigskip

On \'{e}tudie l'\textit{\'{e}quilibre g\'{e}n\'{e}ral} d'une \'{e}conomie
compos\'{e}e de m\'{e}nages (qui travaillent et consomment), d'entreprises
(qui produisent des biens diversifi\'{e}s \`{a} l'aide du facteur travail),
d'un Etat (qui choisit le niveau de la d\'{e}pense publique et l\`{e}ve des
imp\^{o}ts, suppos\'{e}s ici forfaitaires) et d'une banque centrale (qui d%
\'{e}termine le taux d'int\'{e}r\^{e}t r\'{e}el).

\bigskip

\noindent \textbf{Les m\'{e}nages}

Les m\'{e}nages sont tous identiques et \textquotedblleft
ricardiens\textquotedblright\ au sens de la PC\ 1. On suppose que leurs
comportements de demande de consommation ($C_{t}$) et d'offre de travail ($%
L_{t}^{o}$) satisfont les conditions d'optimalit\'{e} suivantes, pour tout $%
t\geq 0$ : 
\begin{equation*}
\frac{C_{t+1}}{C_{t}}=\frac{1+r_{t}}{1+\rho },\ \ \ \frac{\left(
L_{t}^{o}\right) ^{\frac{1}{\xi }-1}}{C_{t}^{-1}}=\frac{W_{t}}{P_{t}},\ \ \
k=0,1,2,...,
\end{equation*}%
avec $r_{t}$ le taux d'int\'{e}r\^{e}t r\'{e}el, $\rho >0$ le taux de pr\'{e}%
f\'{e}rence pour le pr\'{e}sent, $\xi \in \left] 0,1\right[ $ l'\'{e}lasticit%
\'{e} de l'offre de travail, $W_{t}$ le salaire nominal et $P_{t}$ le niveau
g\'{e}n\'{e}ral des prix.

\bigskip

\noindent \textbf{Les entreprises}

Le secteur productif est compos\'{e} d'un continuum d'entreprises index\'{e}%
es par $i\in \left[ 0,1\right] $.  Les entreprises sont détenues par les ménages qui reçoivent leurs profits.
 L'entreprise $i$ produit \`{a} l'aide de la fonction de production $Q_{i,t}=Z_{t}L_{i,t}$, avec $Q_{i,t}$ 
 la quantit\'{e} produite, $L_{i,t}$ la quantit\'{e} de travail utilis\'{e}e et $Z_{t}$
la productivit\'{e} du travail. L'entreprise $i$ est en situation de monopole et
fait face \`{a} la demande :%
\begin{equation*}
Y_{i,t}=Y_{t}\left( P_{i,t}/P_{t}\right) ^{-\eta }
\end{equation*}%
avec $Y_{t}$ la production totale, $P_{i,t}$ le prix nominal du bien $i$ et $%
\eta >1$ l'\'{e}lasticit\'{e} de la demande de bien \`{a} son prix. Comme nous l'avons montr\'{e} au chapitre 4 (eq. 4.4), l'\'{e}quation du prix nominal optimal en log est 
\begin{equation*}
p^{\ast}_{t}=\mu^{\ast} +w_t-z_t
\end{equation*}%
avec $\mu^{\ast}$ le taux de marge optimal. La demande de travail optimal en log est (cf eq 4.6 du poly)
\begin{equation*}
l_{t}^{d}=y_{t}-z_{t}
\end{equation*} 

La productivit\'{e} du travail est constante et normalis\'{e}e \`{a} 
\begin{equation*}
Z_{t}=e^{\xi \mu^{\ast}}
\end{equation*}


\noindent \textbf{L'Etat}

On supposera que la d\'{e}pense publique est nulle \`{a} toutes les p\'{e}%
riodes sauf \`{a} la p\'{e}riode courante (o\`{u} elle est enti\`{e}rement
financ\'{e}e par des imp\^{o}ts forfaitaires) :%
\begin{equation*}
G_{t}>0,\ \ G_{t-1}=G_{t+1}=G_{t+2}=...=0.
\end{equation*}

\noindent \textbf{La banque centrale}

La banque centrale est suppos\'{e}e ne pas r\'{e}agir aux pressions
inflationnistes engendr\'{e}es par la d\'{e}pense publique : elle met en
oeuvre le taux d'int\'{e}r\^{e}t r\'{e}el $r_{t}=\rho $ (le param\`{e}tre $\gamma$ dans la r\`{e}gle PM est z\'{e}ro).

\bigskip

\noindent \textbf{Premi\`{e}re partie : l'\'{e}quilibre OA-DA avec d\'{e}%
pense publique}

\begin{enumerate}
\item Expliquer intuitivement le sens des conditions d'optimalit\'{e} caract%
\'{e}risant le comportement des m\'{e}nages, puis les formuler en log.

\item  Expliquer intuitivement l'\'{e}quation du prix nominal optimal et la demande optimal du travail. 


\item V\'{e}rifier qu'au voisinage de $G_{t}=0$ l'\'{e}quilibre sur le march%
\'{e} des biens donne :%
\begin{equation*}
y_{t}\simeq c_{t}+G_{t},\text{ avec }y_{t}=\ln Y_{t}\text{ et }c_{t}=\ln
C_{t}
\end{equation*}

\item En utilisant les conditions d'\'{e}quilibre sur les march\'{e}s des
biens et du travail, calculer le niveau naturel du produit $y_{t}^{n}$ dans
cette \'{e}conomie et expliquer pourquoi il est influenc\'{e} par la d\'{e}%
pense publique.

\item On suppose que les prix nominaux sont rigides : \`{a} chaque p\'{e}%
riode une fraction $1-\omega $ des entreprises choisit son prix de vente de
mani\`{e}re optimale, alors que les autres font cro\^{\i}tre leur prix de
vente \`{a} un taux \'{e}gal \`{a} l'inflation de la p\'{e}riode pr\'{e}c%
\'{e}dente (cf. chapitre 4). En passant par les m\^{e}mes \'{e}tapes de
calcul que dans le chapitre 4, en d\'{e}duire que la courbe OA en pr\'{e}%
sence de d\'{e}pense publique est donn\'{e}e par :%
\begin{equation*}
\text{\textbf{OA}}:\pi _{t}=\pi _{t-1}+\kappa \left( y_{t}-y_{t}^{n}\right)
,\ \kappa \geq 0,
\end{equation*}%
o\`{u} $y_{t}^{n}$ a \'{e}t\'{e} calcul\'{e} \`{a} la question 4.
\end{enumerate}

\bigskip

\noindent \textbf{Deuxi\`{e}me partie : l'impact d'un choc de d\'{e}pense
publique}

\begin{enumerate}
\item Montrer que la courbe DA de cette \'{e}conomie est donn\'{e}e par : 
\begin{equation*}
\text{\textbf{DA}}:y_{t}=G_{t}
\end{equation*}

\item Calculer analytiquement l'impact sur $y_{t}$ et sur $\pi _{t}$ du choc
de d\'{e}pense publique, et expliquer intuitivement les r\'{e}sultats
obtenus (on supposera que l'inflation \'{e}tait nulle \`{a} la p\'{e}riode pr%
\'{e}c\'{e}dente: $\pi _{t-1}=0$).

\item Repr\'{e}senter dans le plan ($y,\pi $) l'\'{e}quilibre OA-DA et son d%
\'{e}placement suite au choc de d\'{e}pense publique.
\end{enumerate}


\end{document}
