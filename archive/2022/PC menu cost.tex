%2multibyte Version: 5.50.0.2960 CodePage: 1252

\documentclass[11pt,a4paper]{article}
%%%%%%%%%%%%%%%%%%%%%%%%%%%%%%%%%%%%%%%%%%%%%%%%%%%%%%%%%%%%%%%%%%%%%%%%%%%%%%%%%%%%%%%%%%%%%%%%%%%%%%%%%%%%%%%%%%%%%%%%%%%%%%%%%%%%%%%%%%%%%%%%%%%%%%%%%%%%%%%%%%%%%%%%%%%%%%%%%%%%%%%%%%%%%%%%%%%%%%%%%%%%%%%%%%%%%%%%%%%%%%%%%%%%%%%%%%%%%%%%%%%%%%%%%%%%
\usepackage{graphicx}
\usepackage{amsmath}

\setcounter{MaxMatrixCols}{10}
%TCIDATA{OutputFilter=LATEX.DLL}
%TCIDATA{Version=5.50.0.2960}
%TCIDATA{Codepage=1252}
%TCIDATA{<META NAME="SaveForMode" CONTENT="1">}
%TCIDATA{BibliographyScheme=Manual}
%TCIDATA{LastRevised=Tuesday, October 23, 2018 17:52:06}
%TCIDATA{<META NAME="GraphicsSave" CONTENT="32">}

\newtheorem{theorem}{Theorem}
\newtheorem{acknowledgement}[theorem]{Acknowledgement}
\newtheorem{algorithm}[theorem]{Algorithm}
\newtheorem{axiom}[theorem]{Axiom}
\newtheorem{case}[theorem]{Case}
\newtheorem{claim}[theorem]{Claim}
\newtheorem{conclusion}[theorem]{Conclusion}
\newtheorem{condition}[theorem]{Condition}
\newtheorem{conjecture}[theorem]{Conjecture}
\newtheorem{corollary}[theorem]{Corollary}
\newtheorem{criterion}[theorem]{Criterion}
\newtheorem{definition}[theorem]{Definition}
\newtheorem{example}[theorem]{Example}
\newtheorem{exercise}[theorem]{Exercise}
\newtheorem{lemma}[theorem]{Lemma}
\newtheorem{notation}[theorem]{Notation}
\newtheorem{problem}[theorem]{Problem}
\newtheorem{proposition}[theorem]{Proposition}
\newtheorem{remark}[theorem]{Remark}
\newtheorem{solution}[theorem]{Solution}
\newtheorem{summary}[theorem]{Summary}
\newenvironment{proof}[1][Proof]{\textbf{#1.} }{\ \rule{0.5em}{0.5em}}
\oddsidemargin -15pt
\evensidemargin 0pt
\setlength\textwidth{17cm}
\topmargin -30pt
\setlength\textheight{23cm}
%\input{tcilatex}
\begin{document}


\begin{center}
\textbf{Ecole Polytechnique}

\bigskip

\textbf{Eco 432 - Macro\'{e}conomie}

\bigskip

\textbf{PC 3. Co\^{u}ts de catalogue}

\hspace{1.0in}
\end{center}

\bigskip

On consid\`{e}re le mod\`{e}le suivant de concurrence monopolistique avec
rigidit\'{e}s nominales sur le march\'{e} des biens. Les firmes forment un
continuum de longueur 1. Chaque firme maximise son profit r\'{e}el et fait
face \`{a} la fonction de demande%
\begin{equation}
Y_{i}=Y\left( \frac{P_{i}}{P}\right) ^{-\eta },\;\eta >1
\label{Demande relative}
\end{equation}%
o\`{u} $Y_{i}$, $i\in \left[ 0,1\right] $ est la demande adress\'{e}e \`{a}
la firme $i$, $P_{i}$ le prix de vente nominal du bien qu'elle produit, $P$
le niveau g\'{e}n\'{e}ral de prix (qu'on d\'{e}terminera plus loin) et $Y$
la demande agr\'{e}g\'{e}e. Les firmes sont dot\'{e}es de la fonction de
production $Q_{i}=L_{i}$.

L'offre de travail des m\'{e}nages est donn\'{e}e par 
\begin{equation}
L^{o}=A\left( \frac{W}{P}\right) ^{\xi },  \label{Offre de travail}
\end{equation}%
o\`{u} $\xi >0$ est l'\'{e}lasticit\'{e} de l'offre de travail au salaire, $%
W $ le salaire nominal et%
\begin{equation*}
A=\left( \frac{\eta }{\eta -1}\right) ^{\xi }>0
\end{equation*}%
une constante d'\'{e}chelle. 

Enfin, la demande agr\'{e}g\'{e}e est une version simplifiée de celle vue au chapitre 3:%
\begin{equation}
y=\theta-p  \label{Demande agregee}
\end{equation}

où $\theta$ est un choc de demande. 
Comme d'habitude, les lettres minuscules d\'{e}signent le logarithme des
lettres majuscules correspondantes. Par ailleurs, on raisonnera toujours au
voisinage de l'\'{e}quilibre sym\'{e}trique o\`{u} toutes les firmes fixent
le m\^{e}me prix de vente, ce qui implique le log du prix moyen $p=\ln P$
est en premi\`{e}re approximation \'{e}gal \`{a} la moyenne des prix de
vente individuels en log $p_{i}=\ln P_{i}$:%
\begin{equation*}
p\simeq \int_{0}^{1}p_{i}\text{d}i,
\end{equation*}%
et de m\^{e}me pour la demande de travail\ totale en log :%
\begin{equation*}
l^{d}\simeq \int_{0}^{1}l_{i}\text{d}i.
\end{equation*}

\vspace{1cm}

\noindent \textbf{Premi\`{e}re partie : prix de vente optimal et \'{e}%
quilibre naturel}

\bigskip

\noindent \textbf{1.} Interpr\'{e}ter les \'{e}quations du mod\`{e}le.

\bigskip

\noindent \textbf{2.} Calculer le prix de vente optimal de la firme, 
$P^{\ast }$, en fonction du salaire nominal $W,$ et interpr\'{e}ter.


%Montrer qu'au voisinage de $P^{\ast }$ le choix d'un prix de vente $p_{i}\neq
%p^{\ast }$ engendre une perte proportionnelle de profit de l'ordre de $%
%K\left( p_{i}-p^{\ast }\right) ^{2}$, o\`{u} $K$ est une constante positive
%qui d\'{e}pend de l'intensit\'{e} de la concurrence sur le march\'{e} des
%biens.

\bigskip
Dans ce qui suit, exprimez toutes les variables en logarithme.
\bigskip

\noindent \textbf{3.}  Utiliser l'\'{e}quilibre sur le march\'{e} du travail
pour exprimer le prix\ r\'{e}el optimal $p^{\ast }-p$ en fonction de $y$ et
expliquer le r\'{e}sultat obtenu.

\bigskip

\noindent \textbf{4.} Calculer l'\'{e}quilibre de prix flexibles ($%
y^{n},p^{n}$) lorsque $\theta=\theta_{0}=0$ 

\bigskip

\noindent \textbf{Deuxi\`{e}me partie : co\^{u}ts de catalogue h\'{e}t\'{e}%
rog\`{e}nes et pente de l'arbitrage}

\bigskip

On suppose dans cette partie que $\xi =1$, et que l'\'{e}conomie est \`{a} l'%
\'{e}quilibre naturel ($y^{n},p^{n}$) avant un choc de demande de taille $\Delta
\theta=\theta_{1}-\theta_{0}$ ($=\theta_{1}$). Les firmes font face \`{a} des co\^{u}ts fixes de changement de
prix (co\^{u}ts de catalogues), de sorte qu'elles peuvent rationnellement
choisir de maintenir leur prix de vente au niveau $p^{n}$ m\^{e}me apr\`{e}s
le choc. Les co\^{u}ts de catalogue diff\`{e}rent d'une firme \`{a} l'autre
: en classant les firmes $i\in \left[ 0,1\right] $ par ordre croissant de co%
\^{u}t de catalogue, on suppose que la firme $i$ fait face au co\^{u}t :%
\begin{equation*} 
z\left( i\right) =\bar{z}i
\end{equation*}%
o\`{u} $\bar{z}$ est une constante positive.

\bigskip

\noindent \textbf{5.} Le choix d'un prix de vente $p_{i}\neq p^{\ast }$ engendre une perte de profit. Au voisinage de l'\'{e}quilibre sym\'{e}trique (o\`{u} $p^{\ast }\simeq p$) la perte est de second ordre et égale \`{a} $K \left(p_ {i} -p ^ {\ast} \right) ^ {2} $, o\`{u} $ K> 0 $ est une constante positive.\footnote{Il n'est pas nécessaire de dériver ce résultat. Pour les éleves intéressés,  le corrigée montrera que ce résultat est une bonne approximation.} Plus l'écart entre le prix courant  et le prix optimal  est important, plus la perte est importante.  Calculer la proportion $\omega \in \left[ 0,1\right] $
de firmes maintenant leur prix inchang\'{e} en fonction du nouveau prix
optimal $p^{\ast }$ pr\'{e}valant apr\`{e}s le choc, et interpr\'{e}ter.

\bigskip

\noindent \textbf{6.} Calculer le niveau g\'{e}n\'{e}ral des prix $p$ en
fonction de $p^{\ast }$, et en d\'{e}duire la relation entre $p$, $\omega $
et $y.$ Interpr\'{e}ter le r\'{e}sultat obtenu.

\bigskip

\noindent \textbf{7.} Utiliser les r\'{e}ponses aux questions 5 et 6 pour
montrer que la courbe d'offre agr\'{e}g\'{e}e peut s'\'{e}crire%
\begin{equation}
y\left( p\right) =\left( \left( \bar{z}/K\right) ^{\frac{1}{3}}\left\vert
p\right\vert ^{-\frac{2}{3}}-1\right) p,  \tag{OA}
\end{equation}%
et repr\'{e}senter graphiquement cette fonction dans le plan ($y,p$). Commentez la pente de la courbe. 

%Montrer que la taille du choc ($\Delta \theta$) affecte son impact sur le produit
%(d$y/$d$\left( \Delta \theta \right) $) et expliquer pourquoi. Pourquoi les param%
%\`{e}tres ($K,\bar{z}$) influencent-ils la pente de la courbe OA ?

\vspace{1cm}

\noindent \textbf{Troisi\`{e}me partie (facultatif) : co\^{u}ts de catalogue
et multiplicit\'{e} d'\'{e}quilibres}

\bigskip

On suppose maintenant que $\xi >1$, et que toutes les firmes font face au m%
\^{e}me co\^{u}t de catalogue $\bar{z} $. Comme dans la deuxi\`{e}me
partie, l'\'{e}conomie est initialement \`{a} l'\'{e}quilibre naturel et on
s'interroge sur les incitations qu'on les firmes \`{a} changer leur prix
suite au choc $\Delta \theta$.


\bigskip


\noindent 8. Exprimer $p$, $y$ et $p^{\ast }$ en fonction de $\Delta \theta$ et $%
1-\omega $

\bigskip

\noindent 9. Calculer $Kp^{\ast 2}$ en fonction de $\omega $, et interpr\'{e}%
ter le r\'{e}sultat obtenu.

\bigskip

\noindent 10. Calculer les valeurs de $\Delta \theta $ pour lesquelles

\begin{itemize}
\item $\omega =0$ est le seul \'{e}quilibre de Nash sym\'{e}trique

\item $\omega =1$ est le seul \'{e}quilibre de Nash sym\'{e}trique

\item les deux \'{e}quilibres de Nash sont possibles.
\end{itemize}

\newpage



\end{document}

