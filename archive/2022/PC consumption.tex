%2multibyte Version: 5.50.0.2960 CodePage: 1252

\documentclass[a4paper,11pt]{article}
%%%%%%%%%%%%%%%%%%%%%%%%%%%%%%%%%%%%%%%%%%%%%%%%%%%%%%%%%%%%%%%%%%%%%%%%%%%%%%%%%%%%%%%%%%%%%%%%%%%%%%%%%%%%%%%%%%%%%%%%%%%%%%%%%%%%%%%%%%%%%%%%%%%%%%%%%%%%%%%%%%%%%%%%%%%%%%%%%%%%%%%%%%%%%%%%%%%%%%%%%%%%%%%%%%%%%%%%%%%%%%%%%%%%%%%%%%%%%%%%%%%%%%%%%%%%
\usepackage{amsfonts}
\usepackage{graphicx}
\usepackage{amsmath}

\setcounter{MaxMatrixCols}{10}
%TCIDATA{OutputFilter=LATEX.DLL}
%TCIDATA{Version=5.50.0.2960}
%TCIDATA{Codepage=1252}
%TCIDATA{<META NAME="SaveForMode" CONTENT="1">}
%TCIDATA{BibliographyScheme=Manual}
%TCIDATA{Created=Wed Apr 05 06:17:18 2000}
%TCIDATA{LastRevised=Tuesday, October 23, 2018 17:51:02}
%TCIDATA{<META NAME="GraphicsSave" CONTENT="32">}
%TCIDATA{<META NAME="DocumentShell" CONTENT="General\Blank Document">}
%TCIDATA{Language=French}
%TCIDATA{CSTFile=LaTeX article (bright).cst}

\newtheorem{theorem}{Theorem}
\newtheorem{acknowledgement}[theorem]{Acknowledgement}
\newtheorem{algorithm}[theorem]{Algorithm}
\newtheorem{axiom}[theorem]{Axiom}
\newtheorem{case}[theorem]{Case}
\newtheorem{claim}[theorem]{Claim}
\newtheorem{conclusion}[theorem]{Conclusion}
\newtheorem{condition}[theorem]{Condition}
\newtheorem{conjecture}[theorem]{Conjecture}
\newtheorem{corollary}[theorem]{Corollary}
\newtheorem{criterion}[theorem]{Criterion}
\newtheorem{definition}[theorem]{Definition}
\newtheorem{example}[theorem]{Example}
\newtheorem{exercise}[theorem]{Exercise}
\newtheorem{lemma}[theorem]{Lemma}
\newtheorem{notation}[theorem]{Notation}
\newtheorem{problem}[theorem]{Problem}
\newtheorem{proposition}[theorem]{Proposition}
\newtheorem{remark}[theorem]{Remark}
\newtheorem{solution}[theorem]{Solution}
\newtheorem{summary}[theorem]{Summary}
\newenvironment{proof}[1][Proof]{\textbf{#1.} }{\ \rule{0.5em}{0.5em}}
%\input{tcilatex}
\oddsidemargin 0pt
\evensidemargin 0pt
\setlength\textwidth{17.5cm}
\setlength{\topmargin}{-2cm}
\setlength{\oddsidemargin}{-1cm}
\setlength\textheight{25cm}

\begin{document}


\begin{center}
\textbf{Ecole Polytechnique}

\textbf{Eco 432 - Macro\'{e}conomie}

\bigskip

\textbf{PC 4. La demande de consommation}
\end{center}

\bigskip


\noindent \textbf{Exercice : Choix de consommation}

\smallskip

\noindent


On \'{e}tudie les choix de consommation de m\'{e}nages qui maximisent leur
utilit\'{e} intertemporelle. On raisonne sous l'hypoth\`{e}se de pr\'{e}visions parfaites,
de sorte que les m\'{e}nages anticipent parfaitement les valeurs futures de
leur revenu.

Les individus vivent deux périodes. L'utilit\'{e} intertemporelle du m\'{e}nage $j$ \`{a} partir de la p\'{e}riode $t=0$ est
donn\'{e}e :%
\begin{equation}
\mathcal{U}_{0}=\ln
C_{0}^{j} + \left( \frac{1}{1+\rho }\right) \ln
C_{1}^{j},  \label{2 - utilite intertemporelle}
\end{equation}%
avec $C_{t}^{j}$ la consommation du m\'{e}nage et $%
\rho >0$ son \textquotedblleft taux de pr\'{e}f\'{e}rence pour le pr\'{e}%
sent\textquotedblright\ (plus $\rho $ est \'{e}lev\'{e}, moins la
consommation future est valoris\'{e}e relativement \`{a} la consommation pr%
\'{e}sente).

Le m\'{e}nage fait face, \`{a} chaque p\'{e}riode $t=0,1$, \`{a} la contrainte
budg\'{e}taire suivante :%
\begin{equation}
\underset{\text{consommation}}{\underset{\uparrow }{\underset{}{C_{t}^{j}}}}+%
\underset{\text{\'{e}pargne financi\`{e}re nette }S_{t}^{j}}{\underbrace{%
\underset{}{A_{t}^{j}-A_{t-1}^{j}}}}=\underset{\text{revenus du patrimoine}}{%
\underbrace{\underset{}{r_{t-1}A_{t-1}^{j}}}}\ +\ \underset{\text{revenu
salarial }}{\underbrace{\underset{}{\frac{W_{t}}{P_{t}}\bar{L}_{t}^{j}}}}
\label{2 - CB}
\end{equation}%
avec $\bar{L}_{t}^{j}$ est la quantit\'{e} de travail fournie par le m\'{e}%
nage (ici exog\`{e}ne), $A_{t}^{j}$ son patrimoine (en fin de p\'{e}riode $t$%
), $W_{t}/P_{t}$ le salaire r\'{e}el et $r_{t}$ le taux d'int\'{e}r\^{e}t r%
\'{e}el entre les p\'{e}riodes $t$ et $t+1$. $(W_{t}/P_{t})\bar{L}_{t}^{j}$
est donc le revenu salarial du m\'{e}nage. Par la suite, on \'{e}crira le revenu salarial $h^j_t = (W_{t}/P_{t})\bar{L}_{t}^{j}$. Par souci de simplicit\'{e} on
fait ici abstraction des imp\^{o}ts. Le patrimoine initial $A^j_{-1}$  est donné car il est le résultat de décisions passées

Les m\'{e}nages peuvent par ailleurs \^{e}tre sujets \`{a} une contrainte
d'endettement de la forme : 
\begin{equation}
A_{t}^{j}\geq \frac{\bar{D}_{t}}{1+r_{t}}\text{, \ avec }\bar{D}_{t}\leq 0,
\label{2 - CE}
\end{equation}%
de sorte qu'ils ne peuvent s'endetter que jusqu'au point o\`{u} la dette 
\`{a} rembourser (capital et int\'{e}r\^{e}t) est \'{e}gale \`{a} $-\bar{D}%
_{t}$.

\begin{enumerate}
\item Expliquer intuitivement pourquoi $%
A_{1}^{j}=0$ [indice : distinguer les cas $A_{1}^{j}<0$ et $%
A_{1}^{j}>0 $]; en d\'{e}duire la contrainte budg\'{e}taire
intertemporelle du m\'{e}nage :%
\begin{equation}
C^j_0+\frac{C^j_1}{1+r_0}
=A_{-1}^{j}\left( 1+r_{-1}\right) +  h^j_0+\frac{h^j_1}{1+r_0}
\label{2 - CBI}
\end{equation}%
et interpr\'{e}ter cette expression.

%\item Montrer que l'\'{e}quation (\ref{2 - CBI}) se g\'{e}n\'{e}ralise au
%cas o\`{u} $n\rightarrow \infty $ d\`{e}s lors que :%
%\begin{equation*}
%\mathcal{A}_{t}:=\lim_{n\rightarrow +\infty }\frac{A_{t+n}}{\Pi
%_{m=0}^{n-1}\left( 1+r_{t+m}\right) }=0
%\end{equation*}%
%Expliquer intuitivement pourquoi cette condition est satisfaite [indice : l%
%\`{a} encore, distinguer les cas $\mathcal{A}_{t}<0$ et $\mathcal{A}_{t}>0$].

\item Un m\'{e}nage \textit{ricardien} (R) est un m\'{e}nage dont les choix de
consommation ne sont jamais contraints par la contrainte d'endettement (\ref%
{2 - CE}). Ecrire le lagrangien correspondant au probl\`{e}me de
maximisation d'un m\'{e}nage ricardien et en d\'{e}duire que ses choix
satisfont la condition :%
\begin{equation*}
\frac{C_{1}^{R}}{C_{0}^{R}}=\frac{1+r_{0}}{1+\rho },
\end{equation*}%
et interpr\'{e}ter cette relation.

\item  Utiliser les réponses précédentes pour calculer $%
C_{0}^{R}$ et $%
C_{1}^{R}$ en fonction de $\rho $ et de la richesse totale, le côté droit de l'\'{e}%
quation (\ref{2 - CBI}), et expliquer intuitivement l'expression obtenue. Considerons le cas $\rho=r_0$ et $A_{-1}=0$. Étudiez comment $C_{0}^{R} $ et $%
A_{0}^{R}$ changent lorsqu'il y a une augmentation de $ h^j_0 $ (revenu courant) ou de $h ^ j_1$ (revenu futur).

%Appelons $R_{t}=\frac{W_{t}}{P_{t}}\bar{L}_{t}^{j}$ le revenu salarial
%apr\`{e}s imp\^{o}t du m\'{e}nage $j$, et supposons pour cette question que
%le taux d'int\'{e}r\^{e}t r\'{e}el est constant \`{a} $r_{t}=\rho $.
%Calculer les sentiers de consommation $\left\{ C_{t}^{R}\right\}
%_{t=0}^{+\infty }$ et de patrimoine $\left\{ A_{t}^{R}\right\}
%_{t=0}^{+\infty }$ d'un m\'{e}nage ricardien vivant ind\'{e}finiment ($%
%n\rightarrow +\infty $), dont le patrimoine initial est nul ($A_{t-1}^{R}=0$%
%) et dont le sentier de revenu est le suivant $R_{t}=R\geq 0$ de $t=0$ \`{a} 
%$t=T-1$ ($T\geq 1$), puis $R_{t}=R+\Delta R>R$ \`{a} partir de $t=T$.

\item Un m\'{e}nage \textit{keyn\'{e}sien} (K) est un m\'{e}nage dont la
consommation courante est syst\'{e}matiquement contrainte par l'\'{e}quation
(\ref{2 - CE}). Calculer sa consommation, et expliquer pourquoi elles est d%
\'{e}croissante du taux d'int\'{e}r\^{e}t r\'{e}el.

%\item En supposant que l'\'{e}conomie est compos\'{e}e de m\'{e}nages keyn%
%\'{e}siens et ricardiens, en d\'{e}duire qu'on peut \'{e}crire la
%consommation agr\'{e}g\'{e}e sous la forme du fonction de consommation de la
%forme :%
%\begin{equation}
%C_{t}=C(\underset{-}{\bar{D}_{t}},\underset{-}{r_{t}},\underset{+}{Y_{t}},%
%\underset{+}{Y_{t+1}},,...)  \label{3 - fonction de consommation}
%\end{equation}
\end{enumerate}


\noindent \textbf{Exercice : Épargne de précaution}


\bigskip


Considérons le choix de consommation d'un individu qui vit pendant deux périodes, $t=0,1$. Supposons que l'utilité 
à chaque période soit

\begin{equation}
u(C_t)=\left\{ 
\begin{array}{l}
ac -\frac{b}{2} (C_t)^2\text{\qquad  } if \  \ \ \ C_t \in [0,a/b ] \\ 
a^2/(2b) \text{\qquad } C_t \geq a/b  %
\end{array}%
\right.  \label{uti}
\end{equation}
%%sent\textquotedblright\ est $\rho =0$. 

Le revenu de la première période est $ h_0 $. Le revenu de la deuxième période est $ h_1 $. 
À chaque $ t = 0,1 $, la contrainte budgétaire est \begin{equation}
C_{t}+ A_{t}-A_{t-1}=r_{t-1}A_{t-1}  + h_t
\label{2 - CB}
\end{equation}%

On suppose que $ r_0=r_{-1}=0 $ et que le patrimoine initiale soit nul: $ A_{- 1} = 0 $. 
Sachant que 
à l'équilibre les individus choisissent $A_1 = 0 $, les contraintes budgétaires peuvent s'écrire:

\begin{equation}
C_{0} + A_{0}=h_0
\label{2 - CB}
\end{equation}

\begin{equation}
C_{1} = A_{0}+h_1
\label{2 - CB}
\end{equation}


Le revenu de la première période est $h_0=a/b$. Une caractéristique importante de cet exercice est que le revenu de la deuxième période est stochastique: 
\begin{equation}
h_1=\left\{ 
\begin{array}{l}
a/b +\sigma \text{\qquad  } avec  \  \ \ \ \ probabilité \ 1/2 \\ 
a/b -\sigma  \text{\qquad } avec \ \ probabilité \  1/2  %
\end{array}%
\right.  \label{uti2}
\end{equation}

Aujourd'hui ($t=0$), les individus ne savent pas quel sera le revenu de la période suivante, mais on suppose que ils connaissent  la distribution de probabilité (\ref{uti2}). Une augmentation de $ \sigma $ ne change pas la valeur espérée du revenu, mais peut être interprétée comme une augmentation de l'incertitude (``mean-preserving spread"). Plus de dispersion signifie que le revenu est «plus risqué». 





Le consommateur maximise, à la date 0, l'espérance de la somme des utilités futures actualisées. On suppose que $\rho=0$ (les individus sont infiniment patients). En utilisant les contraintes budgétaires, nous écrivons l'utilité intertemporelle espérée: 
\begin{equation}
\mathcal{U}_{0}=\underset{\text{utilité 
aujourd'hui}}{\underbrace{u(h_0-A_0)} } + \frac{1}{2} \underset{\text{utilité 
si $h_1$ est bas}}{\underbrace{u (A_0+\frac{a}{b} -\sigma)} }   + \frac{1}{2} \underset{\text{utilité 
si $h_1$ est élevé}}{\underbrace{u (A_0+\frac{a}{b} +\sigma)} }   \label{2 - utilite intertemporelle}
\end{equation}%

\begin{enumerate}
\item Tracez l'utilité marginale $u'(C_t)$ en fonction de la consommation.

\item Tout d'abord, supposons que $\sigma = 0 $ (aucun risque). Trouvez l'épargne optimale $A_0$; puis, calculez $ C_0 $ et $ C_1 $. Deuxièmement, supposons que le revenu $h_1$ soit incertain.  Étudiez comment $\sigma$ affecte les choix optimaux de  $ A_0 $, $ C_0 $ et $ C_1 $.
Comparez les deux cas ($\sigma=0$ et $\sigma>0$). Y a-t-il plus d'épargne lorsque le revenu est incertain? Expliquez. 

\end{enumerate}
\smallskip



\end{document}
